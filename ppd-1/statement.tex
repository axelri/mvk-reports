\section{Problembeskrivning}
\subsection{Detaljerad problembeskrivning}
\subsection{Motivering}
Alla gruppmedlemmar har ett stort intresse för teknisk utveckling och det är något vi alla vill arbeta med i framtiden. Det projekt vi valde reflekterar våra framtida ambitioner fast på en mildare nivå; att arbeta i grupp med att ta fram en teknisk lösning som skall levereras till ett företag (arbetsgivaren, The Beta Family i vårat fall) inom en given tidsram. Genom att prova på hur det är att jobba i grupp mot ett företag med ett projekt som faktiskt är tänkt att nå marknaden kommer vi att få erfarenheter som vi kommer kunna utnyttja i framtiden.

\subsubsection{Tekniska områden}

Projektet innefattar olika tekniska områden som tillsammans med kursens mål kommer att ge oss en inblick i hur projektarbete går till i arbetslivet. 
\begin{itemize}
	\item Androidprogrammering
	\begin{itemize}
		\item Back-end i form av inspelning av skärm, kamera, ljud och touch samt insamling av telefondata så som modell, skärmstorlek m.m.
		\item Front-end i form av ett intuitivt grafiskt användargränssnitt.
		\item Hur denna programvara kan integreras i appar som ska testas.
	\end{itemize}
	\item Datahantering; hur den insamlade datan från telefonen ska sammanställas, skickas och tas emot av en webbserver för att sedan presenteras i ett webbgränssnitt.
\end{itemize}

\subsubsection{Framtiden inom teknisk utveckling}
Den tekniska utvecklingen och i synnerhet mobiltelefoni har under de senaste åren gått framåt i rasande tempo. Fler och fler företag använder sig utav mobilappar för att nå ut till sina kunder med information, tjänster och produkter. Det är helt klart en fördel att känna till hur utvecklingsprocessen av en app går till när man som nyexaminerad civilingenjör når arbetslivet.

\subsection{Målsättning}
\subsection{Färdigheter och förkunskaper}
