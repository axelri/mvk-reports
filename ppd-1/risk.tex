\section{Riskbedömning}

\subsection{Skärminspelning}
I och med att skärminspelning inte är lika enkelt att utföra på Android som på iOS om man inte har tillgång till root finns risken att vi inte får så många bilder per sekund som vi önskar. TBF har meddelat att om det inte finns någon möjlighet att få en bra bildhastighet så räcker det med att få det lika bra som TestFairy. Bristen på erfarenhet av att skapa en screen-capture gör dock att det finns en risk att prestandan inte blir tillräckligt bra. I och med att detta anses vara en risk har det tagits i åtanke vid skapandet av planeringen.

\subsection{Tidsplanering}
Bristen på erfarenhet i gruppen kan betyda att vår planering inom vissa delområden är optimistiska bedömningar. Ingen i gruppen har någon större erfarenhet av mikrofon-, kamera- eller touchinspelning, men det har förmodats utifrån forskning inom de olika delområdena att dessa är de minst tidskrävande. Vår riskbedömning av skärminspelningen har lett till att vi planerat in mycket tid för delområdet. Skärminspelningsarbetet kan vara flexibelt till viss del då mycket av arbetet kommer bestå av att försöka optimera prestandan. Därmed kan risken med planeringen hanteras i senare delar av projektet om det skulle komma till att planeringen brister.

\subsection{Inspelning av rörelser}
I den version av Screen Recorder som används till iOS syns testarens rörelser i videon. Eftersom iOS-versionen har andra förutsättningar som möjliggör en högre bilduppdateringsfrekvens blir rörelserna naturliga. Om det inte är möjligt att genomföra på Android finns det en risk i hur man vill lösa presentationen av rörelser. Antag att lösningen blir liknande den TestFairy använder, detta innebär i så fall att en bild visas per sekund. Visas bara en punkt där fingret befinner sig på varje bild blir det omöjligt att avgöra hur användaren rör sig runt i applikationen. Det behövs någon form av ``historia'' för rörelserna för att man ska kunna avgöra om testaren drar fingret, trycker eller sveper. Risken ligger inte lika mycket i den tekniska biten som i att gruppen måste komma överens om hur det ska presenteras.

\subsection{Applikationens gränssnitt}
Den befintliga iOS-versionen tar bara ett par minuter för utvecklaren att implementera. För testaren krävs därefter inte heller något större arbete för att spela in. Simpelheten i applikationen är viktigt och måste bevaras för Android-versionen. Några av gruppmedlemmarna har erfarenhet av GUI-utveckling och att uppskatta en rimlig tidsåtgång till delområdet är därmed inte helt omöjligt. Risken består dock av att bristen på erfarenhet inom Android leder till att det tar längre tid än uppskattat och för att hantera detta har arbetet med gränssnittet fått en stor tidsmarginal.