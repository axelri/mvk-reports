\section{Riskbedömning}
\subsection{Ljudinspelning}
En potentiell risk för projektet är inspelningen från mikrofonen då ingen i gruppen har tidigare erharenhet av just detta. Vid närmare undersökning ser det dock inte ut som ett större hinder då ramverket för multimedia stöder ljudinspelning med en rad olika ljudformat med klassen MediaRecorder \parencite{sound} och det finns dessutom en steg-för-steg-guide på Androids utvecklarhemsida. Med gruppens samlade erharenhet av programmering anses därför ljudinspelningen vara en mindre risk. Av denna anledning har tidsåtgången för ljuinspelningen uppskattats till 5 timmar (se avsnitt \ref{sec:projektplan}).

\subsection{Inspelning av rörelser}
I den version av Screen Recorder som används till iOS syns testarens rörelser i videon. Eftersom iOS-versionen har andra förutsättningar som möjliggör en högre bilduppdateringsfrekvens blir rörelserna naturliga. Om det inte är möjligt att genomföra på Android finns det en risk i hur man vill lösa presentationen av rörelser. Antag att lösningen blir liknande den TestFairy använder, detta innebär i så fall att en bild visas per sekund. Visas bara en punkt där fingret befinner sig på varje bild blir det omöjligt att avgöra hur användaren rör sig runt i applikationen. Det behövs någon form av ``historia'' för rörelserna för att man ska kunna avgöra om testaren drar fingret, trycker eller sveper. Risken ligger inte lika mycket i den tekniska biten som i att gruppen måste komma överens om hur det ska presenteras.
