\section{Riskbedömning}

\subsection{Inspelning av rörelser}
I den version av Screen Recorder som används till iOS syns testarens rörelser i videon. Eftersom iOS-versionen har andra förutsättningar som möjliggör en högre bilduppdateringsfrekvens blir rörelserna naturliga. Om det inte är möjligt att genomföra på Android finns det en risk i hur man vill lösa presentationen av rörelser. Antag att lösningen blir liknande den TestFairy använder, detta innebär i så fall att en bild visas per sekund. Visas bara en punkt där fingret befinner sig på varje bild blir det omöjligt att avgöra hur användaren rör sig runt i applikationen. Det behövs någon form av ``historia'' för rörelserna för att man ska kunna avgöra om testaren drar fingret, trycker eller sveper. Risken ligger inte lika mycket i den tekniska biten som i att gruppen måste komma överens om hur det ska presenteras.

\subsection{Skärminspelning}
I och med att skärminspelning inte är lika enkelt att utföra på Android som på IOS om man inte har tillgång till root finns risken att vi inte får så många bilder per sekund som vi önskar. TBF har meddelat att om det inte finns någon möjlighet att få en bra bildhastighet så räcker det med att få det lika bra som TestFairy. Bristen på erfarenhet av att skapa en screen-capture gör dock att det finns en risk att prestandan inte blir tillräckligt bra. I och med att detta anses vara en risk har det tagits i åtanke vid skapandet av planeringen.

\subsection{Kamerainspelning}
En annan teknisk utmaning är att spela in användarens ansikte med enhetens frontkamera. Android tillåter applikationer att spela in via frontkameran genom ett inbyggt bibliotek men eftersom att det så många olika enheter som använder Android är det inte säkert att inställningarna för att spela in från frontkameran blir desamma för alla enheter. Tanken är också att användarens ansikte automatiskt skall kännas igen vid kalibrering vilket kan bli en teknisk utmaning eftersom den inspelade videon måste behandlas i realtid för att känna igen ansikten.