\section{Riskbedömning}

Vi bedömer att de största riskerna för projektet är följande:
(En fullständig Risk Assessment Template finns som appendix)

\subsection{Skärminspelningens prestanda är för låg}
\label{subsec:screenrec}
I och med att skärminspelning inte är lika enkelt att utföra på Android som på iOS om man inte har tillgång till root finns risken att vi inte får så många bilder per sekund som vi önskar. TBF har meddelat att om det inte finns någon möjlighet att få en bra bildhastighet så räcker det med att få det lika bra som TestFairy. Bristen på erfarenhet av att skapa en screen-capture gör dock att det finns en risk att prestandan inte blir tillräckligt bra. Om fallet är sådant kan det betyda att TBFs kunder inte anser det vara värt att betala för testningen vilket skulle innebära att produkten inte platsar i TBFs affärsmodell. Vi har därför gett skärminspelningen mycket plats i planeringen då det är kritiskt att kraven uppfylls. För att se till att detta krav uppfylls kommer utförliga tester på icke-triviala applikationer att göras för att se till att presetanan uppmäter förväntningarna.

\subsection{Tidsplaneringen är för optimistisk}
Bristen på erfarenhet i gruppen kan betyda att vår planering inom vissa delområden är optimistiska bedömningar. Ingen i gruppen har någon större erfarenhet av mikrofon-, kamera- eller touchinspelning, men det har förmodats utifrån forskning inom de olika delområdena att dessa är de minst tidskrävande. Vår riskbedömning av skärminspelningen har lett till att vi planerat in mycket tid för delområdet. Skärminspelningsarbetet kan vara flexibelt till viss del då mycket av arbetet kommer bestå av att försöka optimera prestandan. Därmed kan risken med planeringen hanteras i senare delar av projektet om det skulle komma till att planeringen brister. För att hantera denna risk kommer gruppen lägga stor kraft på att få de kunskaper som krävs. Vi kommer dessutom att ha kontakt med beställaren som tillhandahåller viss expertis.

\subsection{Applikationen blir för ointuitiv}
För testaren krävs inte heller större arbete för att spela in ett test i den befintliga iOS-versionen. Simpelheten i applikationen är viktigt och måste bevaras för Android-versionen för att TBFs kunder ska känna att det är värt att använda produkten. Några av gruppmedlemmarna har erfarenhet av GUI-utveckling och att uppskatta en rimlig tidsåtgång till delområdet är därmed inte helt omöjligt. Risken består dock av att bristen på erfarenhet inom Android leder till att det tar längre tid än uppskattat och för att hantera detta har arbetet med gränssnittet fått en stor tidsmarginal. Vi anser att denna risk försummas eftersom ett av kraven är att gränssnittett ska vara så likt iOS-versionen som möjligt.

\subsection{Videosammansättningen tar för mycket serverresurser}
Videosammansättningen sker på servrar, och uppskattas ta mellan 0.75n och n sekunder för en video av längd n. Detta kan vara ohållbart i en miljö med flera simultana användare. Vi har diskuterat serverkapacitet med kunden och ska jobba för att optimera sammansättningen. För att hantera denna risk kommer vi att begränsa insamlad data (utan att påverka gestaltning) och optimera koden på servern för att på så sätt inte förlora prestanda. Slutligen kommer tunga servertester att genomföras för att testa gränserna för vad servern klarar av.

\subsection{Videosammansättningen är för svår att implementera}
En annan risk som rör videosammansättningen är själva implementationen av den. Då gruppen inte har så stor erfarenhet av videosammansättning kan genomförandet bli svårt, och vi kanske inte kommer lyckas optimera sammansättningen tillräckligt mycket. Vi kommer i första hand fokusera på att implementera en fungerande videosammansättning, för att sedan om vi får tid se över optimeringsmöjligheter. För att minimera denna risk har mycket tid avsatts till denna del.

\subsection{Testning}
Det är inte säkert att gruppen kommer hinna hitta utomstående testare för SuperRecordern. Därför kommer de flesta tester förmodligen ske inom gruppen och kanske inte ge tillräckligt bra feedback. Men vår kund är ett företag som erbjuder testning av appar och bör därför kunna genomföra användartester på sina egna testare. Vi ska undersöka möjligheterna till samarbete på den här punkten. Vi kommer att göra utförliga tester på icke-triviala applikationer utifrån de user case som presenterats för att se till att applikationen fungerar korrekt.

\subsection{SDK:t tar för lång tid att integrera}
Den befintliga iOS-versionen tar bara ett par minuter för utvecklaren att implementera i appen som ska testas. Det har under projektets gång visat sig att det kommer bli mycket svårare med Android-versionen. Gruppen har flera olika idéer för hur implementationen ska bli lättare, men alla är ganska avancerade och kommer ta mycket tid att genomföra. Vi har bestämt oss för att inte prioritera detta utan ta det i mån av tid i slutet av projektet. Denna risk är beställaren införstådd med och vi har deras bekräftelse (från en av deras områdesexpterer) på att det inte går att integrera SDK:t på kort tid, det tar helt enkelt lång tid. 
