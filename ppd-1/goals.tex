\subsection{Målsättning}
Marknaden för mobila enheter växer och med det ökar behovet av kompetens inom området. En viktig aspekt inom detta område är mjukvaruutveckling vilket har hög relevans för oss som datateknologer. Detta är inte bara nödvändigt för de som är intresserade av att etablera sig som mjukvarukonstruktörer utan gynnar även dem som arbetar inom projekt som innefattar mobilapplikationer till någon del. En grafisk designer måste exempelvis vara väl medveten om begränsningarna för alla de operativsystem för smarta telefoner som produkten skall stödja när denne utformar de illustrationer som sedan skall programmeras. Detta betyder att alla medlemmarna kommer att kunna dra karriärsmässig nytta genom detta projekt oavsett vilken väg inom branschen som väljs eftersom risken är stor att nästintill alla någon gång kommer att arbeta med något relaterat till applikationer för smarta telefoner.

Eftersom projektets omfattning tillsammans med medlemmarnas antal gör det omöjligt att tillämpa en strategi där alla tar ansvar projektets helhet görs detta till ett stort sammarbete. Inom gruppen har vi varierande erfarenhet av projekt i större skala. Detta projekt är en ypperlig möjlighet att utöka våran erfarenhet inom mjukvarukonstruktion i grupp. Detta leder dessutom till att design av systemarkitektur blir centralt vilket ytterligare är en färdighet som kommer att slipas genom detta projekt.

Genom utformningen på kursen kommer även lärdom om en professionell arbetsprocess förmedlas till oss som kursdeltagare. Detta innefattar många delar såsom rapportskrivningen, men även att från en specifikation hitta användarkrav för produkten och sedan specificera produkten i detalj och sedan implementera denna.

Slutligen fås det inte glömmas att en gemensam uppfattning inom gruppen är att projektet ska vara en rolig upplevelse tillsammans som en driven och ambitiös grupp. Målet är även att avsluta projektet tillsammans som en bättre grupp och känna att vi har utvecklats genom något vi fann roligt att göra tillsammans. 
