\section{Bakgrund}
\subsection{Kommersiell bakgrund}
På grund av begränsningar i operativsystemet är det relativt få som försökt sig på att lansera en skärminspelare till Android. Det är först i den senaste versionen av Android som möjligheten att spela in vad som händer på skärmen finns och då endast i utvecklingssyfte \parencite{kitkat}. Det krävs att mobilen är kopplad till en dator via USB och detta är främst tänkt att användas för att göra instruktionsfilmer och dylikt.

\subsubsection{TestFairy}
\label{testfairy}
TestFairy är ett företag med en affärsidé som påminner om den The Beta Family har. Målet är att underlätta betatestning av Androidapplikationer. Genom att använda sig av deras lösning kan man få en video som visar hur användaren navigerar i appen, grafer över diverse intressanta data (Exempelvis minnes- och CPU-prestanda), samt en mängd metadata.

Problemet med att Android inte har stöd för skärminspelning löser TestFairy genom att ta en skärmdump en gång i sekunden. Detta resulterar dock i en video som snarare påminner om ett bildspel än en rullande video. Dessutom förlorar man information om hur användaren reagerar på ljud från appen. Den visar dock det väsentliga användaren håller på med och ger i det stora hela en god överblick över hur appen används.

Jämfört med The Beta Family verkar TestFairy vilja fokusera mer på den tekniska aspekten av en app. Fokus ligger på att finna  prestandaproblem snarare än att hitta problem i människa-dator-interaktionen. 

TestFairys betatestning går till som så att man laddar upp sin .apk-fil\footnote{Installationsfilen för applikationen}, väljer vilka testare man vill skicka ut till och sedan väntar man på resultatet. Detta underlättar för utvecklaren som inte behöver oroa sig över hur man implementerar TestFairys SDK\footnote{The Beta Familys iOS-testning förlitar sig på att utvecklaren själv implementerar deras SDK} men en nackdel är att TestFairy inte har en krets med testare. The Beta Family har i dagsläget cirka 9 000 testare medan man i TestFairys fall får förlita sig på att man känner tillräckligt många vänner som vill testa appen.

\subsubsection{Diverse appar som kräver root-access}
\label{rootacc}
Det finns en mängd appar som möjliggör inspelning av mobilskärmen men dessa kräver root-access. Att skaffa root-access är något som kräver en viss kunskap och innebär en viss säkerhetsrisk då man låser upp hela mobilen, ungefär som ett adminkonto i Windows. Detta innebär att dessa appar är svåra att använda i betatestningssyfte då man som företag inte kan kräva att alla betatestare rootar sina mobiler. Dessutom är det oftast mer avancerade användare som rootar sina mobiler och i betatestning är sannolikheten stor att man inte bara vill ha återkoppling från avancerade användare. 

Eftersom dessa appar inte primärt är tänkta för betatestning går det inte uttala sig om deras lämplighet i sådana situationer. Appar har i regel inte tillåtelse att ta bilder av andra appars vyer så en app för betatestning skulle inte fungera utan det krävs någon form utav SDK. Apparna utför en delmängd (skärminspelning) utav det som detta projekt syftar genomföra men vissa delar (röstinspelning och kamerainspelning) saknas.

 Exempel på dessa appar är \emph{SCR Screen Recorder} \parencite{scr} och \emph{Screen Recorder} \parencite{sr}

\subsection{Teknisk bakgrund}
\label{subsec:tekniskbakgrond}
Målet med projektet är att utveckla en SDK åt Androidapplikationer och ett krav från arbetsgivaren är att koden ska vara ``native'', vilket innebär att utvecklingen kommer ske främst i Java, i Androids SDK. 

\subsubsection{Androidfragmentering}
Androidfragmentering syftar till det faktum att Androidplattformen körs i olika versioner på hundratals olika hårdvaruspecifikationer, och då måste skillnader i exempelvis skärmstorlek och tillgängliga bibliotek tas i hänsyn under utvecklingen. Ett av målen med projektet är just att ge applikationsutvecklarna möjligheten att fastställa att deras applikationer faktiskt fungerar som de är tänkta på de olika versionerna av plattformen. Under utvecklingsfasen för detta projekt kommer testning framförallt att utföras på emulatorer som bl.a. ingår i Androids SDK, eller som det finns separat programvara för, exempelvis Genymotion som har en gratis licens. I skrivandets stund (2014-02-15) är KitKat (version 4.4) den senaste Androidversionen, men 20\% av enheterna i cirkulation använder fortfarande Gingerbread (version 2.3.7).\parencite{androidversions} Endast 1,3\% använder en lägre version och Gingerbread är också den lägsta versionen som stöds av Genymotion. Allt detta har lett till beslutet att Gingerbread är den lägsta versionen som detta projekt kommer att stödja. 

\subsubsection{Application Sandboxing}
\label{subsubsec:sandbox}
Androidarkitekturen implementerar ``application sandboxing'' vilket innebär att olika applikationer inte kan interagera med varandra direkt.\parencite{sandbox} Av denna anledning är det en SDK som behöver utvecklas som sedan kan integreras med applikationen som behöver testas. En fristående applikation skulle inte ha tillräckligt mycket behörighet för att spela in skärmen på den applikationen som ska testas. Denna sandboxing medför även att tredjepartens applikationer som tangentbord och Google Maps inte kan spelas in. 
Android har denna arkitektur på grund av säkerhetsskäl. En applikation som exempelvis har tillgång till tangentbordet skulle kunna registrera allt användaren skriver, vilket potentiellt skulle kunna inkludera lösenordet till dennes internetbank. 

\subsubsection{Metadata}
\label{subsubsec:Metadata}
I projektet ingår att samla in metadata om den mobiltelefon som testaren använder. Specifikt ingår i kravlistan att ta reda på testarens telefonmodell och skärmstorlek, samt kontinuerligt skicka med testarens skärmorientering. Det finns goda möjligheter att enkelt begära fram dessa data ur det officiella Android-API:t genom modulerna \textit{Build}, \textit{view.WindowManager} och \textit{app.Activity}. Åtkomst till dessa kräver endast grundläggande utvecklarbehörigheter \parencite{adoc}.

\subsubsection{Datapresentation}
\label{subsubsec:Datapresentation}
Insamlade data ska presenteras som ett flöde av information som gestaltar testarens användarmönster. Upprepade skärmdumpar ska tillsammans med testarens fingeravtryck bli till en video. Tillsammans med kamera- och ljudupptagning på testaren ska beställaren av testningen enkelt kunna bilda sig en uppfattning om hur produkten används.

Androids officiella API inkluderar en klocka med en upplösning på nanosekunder, \textit{System.nanoTime()} \parencite{adoc}. Klockan borde räcka gått och väl för att synkronisera kamera- och ljudupptagning med fingeravläsning och skärminspelning. Med samma klocka kan man trivialt spara tidsdata för varje skärmdump som tas, samt för varje fingerrörelse testaren gör. Man kan då representera testarens handlingar genom att ``måla över'' skärmdumparna med linjer som representarer fingerrörelser. Denna typ av enklare bildbehandlingsverktyg finns inbyggd i Android, i modulen \textit{Canvas} \parencite{adoc}. Dessa ihopslagna bilder kan sedan konverteras till en video med hjälp av den inbyggda modulen \textit{media.MediaCodec} \parencite{adoc}. Skulle det visa sig att \textit{media.MediaCodec} inte är kraftfull nog för våra ändamål finns även det fria alternativet \textit{ffmpeg} tillgängligt på Android \parencite{roman10}.

\subsubsection{Ljudinspelning}
\label{subsubsec:ljudinspelning}
Ramverket för multimedia stöder ljudinspelning med en rad olika ljudformat med klassen \textit{MediaRecorder}. Ljudformat och codecs som stöds är AAC LC, HE-AACv1, HE-AACv2, AAC ELD, AMR-NB, AMR-WB, FLAC, MP3, MIDI, Vorbis och PCM/WAVE \parencite{sound}.

\subsubsection{Kamerainspelning}
\label{subsubsec:kamerainspelning}
Tanken med kamerainspelning är att spela in användarens uttryck när denne testar applikationen. Android tillåter applikationer att spela in via frontkameran genom klassen \texit{MediaRecorder} \parencite{frontcamera} som nämnts ovan. Frontkameran har ett id-nummer som man spelar in ifrån.

\subsubsection{GUI}
\label{subsubsec:gui}
GUI för applikationen är mycket enkel och arbetsgivarnas önskan är att den ska vara så lik iOS-versionen som möjligt. Dubbeltryck med två fingrar drar ner en meny varifrån användaren kan välja att påbörja eller avsluta en inspelning, samt välja inställningar. Bland annat har användaren möjligheten att stänga av kamera- och ljudinspelning och trycksensorens färg kan väljas här. 

Ett alternativ för att skapa menyn är att använda en \textit{dialog}, vilket är et popup-fönster som går att skräddarsy \parencite{dialog}. 

Android har inte en inbyggd funktion för registrering av dubbeltryck med två fingrar, men klassen \textit{MotionEvent} samlar tillräckligt mycket information för att man ska kunna skriva en egen klass som kan avgöra om ett sådant dubbeltryck har skett \parencite{dbletap}.

% TODO: Animation

\subsubsection{Back- och front-end}
\label{subsubsec:backfront}
En back-end behövs för att ta emot datan som generas efter att en användare har betatestat en applikation och en front-end behövs för att presentera denna data för utvecklarna. Arbetsgivarna The Beta Family har redan en iOS-version av denna SDK med tillhörande back-end och front-end. Deras rekommendation är att det här projektet ska ha en egen miljö så att det inte uppstår ett beroende av deras kod, då data som generas i Android är annorlunda mot den som genereras i iOS. Dock bör samma ramverk som deras system är byggda på användas för att förenkla en senare integrering. Dessa ramverk är:


Back-end:
\begin{itemize}
	\item Ramverk: Laravel
	\item Språk: PHP, MySQL, JavaScript
\end{itemize}

Front-end:
\begin{itemize}
	\item Ramverk: Zurb
	\item Språk: HTML, CSS, JavaScript
\end{itemize}