\definecolor{Red}{rgb}{0.9,0.1,0.1}

\section{Projektplan}
\label{sec:projektplan}
Till och med vecka 12 ligger två relativt tunga kurser parallellt med projektet. Därför kommer mindre tid kunna läggas på projektet då. Fokus läggs på rapporterna PPD1 och URD1 men även de lättare delarna av koden kommer färdigställas under den här tiden. Innan vecka 12 kan 10 timmar per person och vecka läggas på projektet, och efter vecka 12 beräknas 20 timmar per person och vecka kunna läggas, de sista veckorna möjligen mer.

Vecka 13 och framåt kommer mycket mer tid läggas på koden. TIll att börja med ska skärminspelningen vara färdig till den 6/4. Sedan måste det jobbas fram ett sätt att komprimera media, och ett mindre backend-system att överföra inspelat data till. Till sist behövs en transfer-lösning. De sista punkterna kommer nog gå in i varandra en del, så de har alla samma deadline, 4/5, då all kod ska vara färdigställd. 

Därefter är det dags att presentera och lämna in rapporterna ADD, PPD2 och URD2. Rapporterna kommer ha påbörjats parallellt med kodandet i April, därför kommer det inte krävas så mycket jobb för att färdigställa dem. Dessutom får de med arbetsprocessen om de skrivs kontinuerligt under projektets gång. Eftersom de ska spegla slutresultatet kommer de inte kunna bli klara innan projektet är klart. 

Deadline för all kod läggs den 4/5. Deadline för rapporterna och presentationsförberedelser läggs den 9/5. 

\begin{figure}[H]
\centering
\begin{tabular}{ | l | l | l |}
  \hline
  \textbf{Deliverable} & \textbf{Deadline} & \textbf{Uppskattad tidsåtgång} \\ \hline
  PPD-1 & v.8 (19/2) & 100h \\ \hline
  URD-1 & v.10 (5/3) & 100h \\ \hline
  ADD-1 & Kursens slut & 150h \\ \hline
  PPD-2 & Kursens slut & 50h \\ \hline
  URD-2 & Kursens slut & 50h \\ \hline
\end{tabular}
\caption*{\textit{Deliverables}}
\end{figure}

\begin{figure}[H]
\centering
\begin{tabular}{ | l | l | l |}
  \hline
  \textbf{Systemmodul} & \textbf{Datum} & \textbf{Uppskattad tidsåtgång} \\ \hline
  Mikrofon & 16/3 & 5h \\ \hline
  Kamera & 16/3 & 10h  \\ \hline
  Touch & 16/3 & 10h \\ \hline
  GUI & 16/3 & 50h \\ \hline
  Skärminspelning & 6/4 & 300h \\ \hline
  Media & 4/5 & 300h \\ \hline
  Transfer & 4/5 & 200h \\ \hline
  Backend & 4/5 & 200h \\ \hline
\end{tabular}
\caption*{\textit{Utveckling}}
\end{figure}

\begin{figure}[H]
\centering
\begin{tabular}{ | l | l | l | }
  \hline
  \textbf{Aktivitet} & \textbf{Deadline} & \textbf{Uppskattad tidsåtgång} \\ \hline
  Möten & - & 260h \\ \hline
  Mötesprotokoll & - & 50h \\ \hline
  Demo & Början av Maj & 50h \\ \hline
\end{tabular}
\caption*{\textit{Övrigt}}
\end{figure}

\subsection{Veckoplanering}

\begin{tabular}{ | p{65pt} || p{110pt} | p{110pt} | p{110pt} |}
  \hline
  Vecka: & 4 & 5 & 6 \\ \hline
  Antal timmar: & 100 & 60 & 70 \\ \hline
  Per person: & 10 & 6 & 7 \\ \hline
  Händelser: & Föreläsningar & Föreläsning & Föreläsning \\ \hline
  & Möte & Möte & Möte med The Beta Family  \\ \hline
  & Val av projekt & Fick projekt & Lunchmöte\\ \hline
  & Skapa infrastruktur & Research om The Beta Family & Genomförbarhetsresearch\\ \hline
  & Google Drive &  & \\ \hline
\end{tabular}

\begin{tabular}{ | p{65pt} || p{110pt} | p{110pt} | p{110pt} |}
  \hline
  Vecka: & 7 & 8 & 9  \\ \hline
  Antal timmar: & 110 & 100 & 100\\ \hline
  Per person: & 11 & 10 & 10\\ \hline
  Händelser: & Föreläsning & Möte & Möte\\ \hline
  & Möte & Sammanfogning av PPD-1 &\\ \hline
  & Uppdelning av PPD-1 & Presentationsförberedelse &  \\ \hline
  & Arbete med PPD-1 & Inlämning av PPD-1 & Arbete med URD-1 \\ \hline
  & Sätta igång med Trello & Arbete med kodgrunden & Arbete med kodgrunden \\ \hline
& & Uppdelning av URD-1 & \\ \hline
\end{tabular}

\begin{tabular}{ | p{65pt} || p{110pt} | p{110pt} | p{110pt} |}
  \hline
  Vecka: & 10 & 11 & 12  \\ \hline
  Antal timmar: & 100 & 100 & 100 \\ \hline
  Per person: & 10 & 10 & 10 \\ \hline
  Händelser: & Möte & Möte & Möte\\ \hline
  & Sammanfogning av URD-1 & Färdigställa kodgrunden & Arbete med skärminspelning\\ \hline
  & Presentationsförberedelse &  &  \\ \hline
  & Inlämning av URD-1 &  &  \\ \hline
  & Arbeta med kodgrunden &  & Arbete med URD-2 \\ \hline
\end{tabular}

\begin{tabular}{ | p{65pt} || p{110pt} | p{110pt} | p{110pt} |}
  \hline
  Vecka: & 13 & 14 & 15  \\ \hline
  Antal timmar: & 200 & 200 & 150 \\ \hline
  Per person: & 20 & 20 & 15 \\ \hline
  Händelser: & Arbete med skärminspelning & Färdigställa skärminspelning & Mediaprogrammering \\ \hline
  &  &  & Backendprogrammering \\ \hline
  &  &  & Transferprogrammering \\ \hline
  &  &  & Arbete med ADD \\ \hline
  &  &  &  \\ \hline
  & Arbete med URD-2 & Arbete med URD-2 & Arbete med URD-2\\ \hline
\end{tabular}

\begin{tabular}{ | p{65pt} || p{110pt} | p{110pt} | p{110pt} |}
  \hline
  Vecka: & 16 & 17 & 18  \\ \hline
  Antal timmar: & 200 & 200 & 300 \\ \hline
  Per person: & 20 & 20 & 30 \\ \hline
  Händelser: & Möte & Möte & Möte \\ \hline
  & Mediaprogrammering & Mediaprogrammering & Deadline för all kod \\ \hline
  & Backendprogrammering & Backendprogrammering &  \\ \hline
  & Transferprogrammering & Transferprogrammering &  \\ \hline
  & Arbete med ADD & Arbete med ADD & Arbete med ADD \\ \hline
  & Arbete med URD-2 & Arbete med URD-2 & Arbete med URD-2\\ \hline
\end{tabular}

\begin{tabular}{ | p{65pt} || p{110pt} | p{233pt} |}
  \hline
  Vecka: & 19 & 20 \\ \hline
  Antal timmar: & 250 & 30 \\ \hline
  Per person: & 25 & 3 \\ \hline
  Händelser: & Möte & Presentation\\ \hline
  & Förbereda DEMO-presentation &\\ \hline
  & Färdigställa ADD &\\ \hline
  & Färdigställa PPD-2 &\\ \hline
  & Färdigställa URD-2 &\\ \hline
\end{tabular}

I schemat står alla aktiviteter som ska utföras under veckan. Vissa aktiviteter som utförs varje vecka står inte med, dessa är mötesprotokollsskrivning och rapportstyrning som utförs av William, projektstyrning som utförs av Paulina och styrning av programmeringsarbetet som utförs av Daniel. Det kommer med största sannolikhet även tillkomma mindre uppgifter. På grund av detta innebär inte de uppskattade timmarna i veckoschemat att alla i gruppen kommer att arbeta med just de uppgifter som står i schemat hela den tiden. Troligt är dock att de utan specifika ansvarsområden kommer ägna nästan all sin planerade tid åt de schemalagda aktiviteterna. 

Det finns inte så mycket utrymme för projektet att ta längre tid än planerat. Men tiden som krävs för varje delmoment är uppskattad med god marginal, så det är sannolikt att projektet kommer hamna före schemat. Det finns även utrymme att spendera mer tid per vecka än planerat om det skulle behövas. Projektledaren kommer kontinuerligt att följa upp och justera tidsplaneringen så ingenting blir försenat. Det blir Chief Programmer’s uppgift att sätta upp delmål i programmeringsdelen och följa upp dessa. Projektledaren och Chief Programmer håller nära kontakt.
