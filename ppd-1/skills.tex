\subsection{Färdigheter och förkunskaper}
Kunskapsmässigt är androidprogrammering den största utmaningen. Eftersom det inte finns någon expertis inom området i gruppen är androidprogrammering, till den nivå som krävs, kunskap som saknas. En stor del av projektet är därmed att många medlemmar i gruppen måste lära sig detta. Fördelaktigt är att språket för android är i java och xml, och gällande java har alla i gruppen stor erfarenhet av språket. För android erbjuds väldigt mycket webbaserad information om allt från steg för steg-pedagogik till specificerad detaljnivå för enskilda implementationer och funktionaliteter. Dessutom finns det mångfallet av forum och dylikt där lösningar för specifika problem diskuteras och presenteras. Allt detta är en fantastisk stor mängd av material att studera för att nå den kunskap som behövs för att genomföra projektet.

Utöver de tekniska färdigheterna finns även sociala erfarenheter i form av ledarskap, större programmeringsprojekt, grafiska gränssnitt genom Swing, och journalistik.  Ledarskapserfarenheterna är mycket relevanta i möjliggörandet av effektivitet och organisation i gruppen. Erfarenheterna av stora programmeringsprojekt lämpar sig väl när systemarkitektur och API ska designas samt att dela upp programmeringen och samordna denna. Erfarenhet med grafiska gränssnitt som skrivits i java är en tillgång för att kunna implementera eventuella gränssnitta för projekten. Journalistiken kommer väl till pass i dokumenteringen av projektet samt i rapportskrivningarna. 

Eftersom det är en möjlighet att projektet även kommer att omfatta serverkommunikation är det relevant att det finns erfarenhet inom PHP och SQL i gruppen. Övriga kunskaper som inte är direkt förknippade med projektet är HTML, CSS, Ruby och javascript.
