\subsection{Begreppsdefinitioner}

\paragraph{Applikation} Körbart program i smartphones, förkortas ibland till ``app''.

\paragraph{Applikationsutvecklare} Person som utvecklar applikationer, förkortas  ibland till ``apputvecklare''.

\paragraph{Bitmap} Grupp lagringsformat för digitala biler där ingen data går förlorad. Lämpad för exempelvis skärmdumpar.

\paragraph{FPS} Frames per seconds, bilder per sekund. Vanligt mått på bildhastighet i video.

\paragraph{GUI} Graphical User Interface är det gränsnitt som en användare av ett program ser och interagerar med.

\paragraph{iOS} Namnet på det operativsystem som används på Apples smartphones.

\paragraph{Laravel} Ett ramverk byggt främst i PHP för att förenkla utvecklandet av skalbara webbapplikationer\parencite{laravel}.

\paragraph{PHP} PHP: Hypertext Preprocessor (\textit{rekursiv akronym}) är ett programmeringsspråk som körs på webbservern för att hantera dynamiskt innehåll på webbplatser.

\paragraph{Protokoll} Samordnad överenskommelse mellan utvecklare av separata system om hur data skall skickas för att möjliggöra korrekt mottagande av data.

\paragraph{SDK} Software Development Kit är en mängd verktyg som gör det möjligt för en utvecklare att bygga en viss mjukvara. Det kan röra sig om allt från ett enskilt bibliotek som tillhandahåller en mängd funktioner till en komplett utvecklingsmiljö.

\paragraph{TBF} The Beta Family, vår projektbeställare.

\paragraph{Zurb} Ett ramverk byggt i HTML, CSS och JavaScript för att underlätta front-end-utveckling hos webbapplikationer\parencite{zurb}.