\subsection{Produktperspektiv}

The Beta Family har som affärsidé med sin produkt Super Recorder att bistå med testare när utvecklare önskar att betatesta sin iOS-applikation genom att sammanföra dessa genom en hemsida. Företaget betalar testarna en belöning och The Beta Family tar en andel av denna. Det samfund The Beta Family byggt upp gör att de kan erbjuda en mer sömlös betatestning jämfört med konkurrenter. 

En brist i tjänsten är att den endast riktar sig till användare som har en smart telefon med iOS och utvecklare som utvecklar produkter för iOS. I dagsläget produceras omkring 80 % av alla smarta telefoner installerade med Android\footnote{ http://www.idc.com/getdoc.jsp?containerId=prUS24442013}. Detta betyder att Super Recorder inte når ut till en anmärkningsvärd mängd potentiella kunder, både bland testare och utvecklare.

Projektet som fångas in av denna rapport har som uppgift att nå de kunder som använder Android. Detta genom att utveckla en SDK för Android som efterliknar den som The Beta Family tillhandahåller iOS så nära som möjligt utefter möjligheterna för Android. Detta möjliggör en bredare testning från utvecklare samt öppnar möjligheterna för testare som har Android i sina smarta telefoner. Projektet utgör därmed en möjlighet att bredda den befintliga tjänsten. En breddning av tjänsten ökar kvantiteten av testare som kan erbjudas, vilket kan vara inbjudande för fler utvecklare – en progression som i slutändan ökar intäkterna från tjänsten.

Eftersom projektet handlar om en efterlikning har inga behov för SDKn i sig behövts sökas då de har tillhandahållits direkt av The Beta Family genom deras SDK för iOS. Däremot har undersökningar i Androids möjligheter och dess begränsningar gjorts för att kunna säkra vad som är genomförbart. Dessa undersökningar gjordes genom att läsa igenom officiella dokument för Android samt liknande problemställningar andra utvecklare stött på och hur dessa lösningar diskuterats.

Eftersom efterlikning är av hög prioritet är följande funktionalitet hos tjänsten ytterst relevant eftersom stöd för Android ska kunnas implementeras på tjänsten utan att systemet ska behöva förändras till en komplicerad och tudelad process för användare. Den nuvarande processen för kunder beskrivs nedan.

Att som utvecklaret anpassa sin applikation för möjliggörandet av nödvändig inspelning av betatestingsessioner är snabb och simpel. Detta moment inkluderar endast att kopiera in några bibliotek till applikationen samt lägga till ett fåtal kodrader som möjliggör visningen av GUI;t\footnote{https://thebetafamily.com/superrecorder/installation}. Detta är en viktig del i tjänsten ty detta leder till att mycket låg arbetsbörda läggs på utvecklaren. Utöver detta behövs inget beaktande från utvecklaren i hur själva videon produceras eller sedan skickas iväg till The Beta Familys servrar då detta sköts av de medföljande biblioteken. 

För testare av en given applikation behövs inga särskilda förberedelser göras. En testare laddar ner applikationen från Super Recorder-hemsidan, startar applikationen, startar inspelningen och följer sedan instruktionerna som utvecklaren har givit testarna. 

Utöver att försöka att återskapa samma funktionalitet i SDKn för Android behövs även integration med The Beta Familys servrar göras. Eftersom det med stor sannolikhet är omöjligt att erhålla identisk data från Android likt den från iOS på grund av deras skillnader kommer inte det nuvarande protokollet mellan servrarna och SDKn att kunna återvinnas i sin helhet hos SDKn för Android. SDKn projektet avser att utveckla är dock inte beroende av något bestämt protokoll vilket möjliggör att projektgruppen själva kan välja det protokoll som ska användas mellan SDKn och servrarna.

Behovet av denna SDK är i stor betydelse för framtida vinster och en ökad mängd kunder. Då smarta telefoner med Android som tidigare nämnt står för en stor del av tillverkade produkter är relevansen att kunna nå ut till en sådan stor mängd potentiella kunder en starkt rekommenderad, möjligtvis även nödvändig, för tjänstens Super Recorders framgång.
