\subsection{Överblick över dokumentet}

Sektion \ref{sec:description} (Generell beskrivning) beskriver olika delar i projektet utifrån ett tekniskt och ett användarmässigt perspektiv.
 
Under \ref{subsec:perspective} (Produktperspektiv) beskrivs idén bakom SuperRecordern och i vilket omfång denna rapport bidrar till utvecklingen av produkten, både genom implementationen av ett Android-SDK men också integrationen med webbplatsen. Under \ref{subsec:generalcapabilities} (Generella förmågor) beskrivs de förmågor som produkten bör ha utifrån applikationsutvecklarens och betatestarens perspektiv. \ref{subsec:constraints} (Begränsningar) handlar om de begränsningar och hinder som kan tänkas uppstå under utvecklandet av produkten. Det handlar främst om tekniska begränsningar inom rörelsehändelser, skärminspelning och överföring av media. En utförligare beskrivning av applikationsutvecklaren och betatestaren finns under \ref{subsec:userdesc} (Användarbeskrivning). Sektionen innehåller även personas (fiktiva representationer av användargrupperna)  med tillhörande scenarion. Under rubriken \ref{subsec:assumptions} (Antaganden och beroenden) diskuteras antaganden kring målgruppen och produkten utifrån ett tekniskt perspektiv. De miljöer som projektet kommer att använda beskrivs ur ett tekniskt perspektiv under \ref{subsec:environment} (Operativ miljö).

Under \ref{subsec:funcreq} (Funktionsmässiga krav) beskrivs hur produkten ska fungera och upplevas av användaren. Kraven (skärm-, röst- och kamerainspelning med flera) listas i tabellform där varje krav är specificerat enligt: beskrivning, motivering, behov, prioritet, källa och verifiering. Liknande struktur hittas under \ref{subsec:techreq} (Tekniska krav och begränsningar) fast där fokuserat på det tekniska så som implementation och testning.