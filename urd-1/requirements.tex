\section{Specifika Krav}

\subsection{Funktionsmässiga krav}

\subsubsection{Skärminspelning}
\begin{tabular}{ | p{65pt} | p{300pt} |}
  \hline
  Identifikator &
Skärminspelning
\\ \hline
  Beskrivning & 
Allt som användaren ser ska dokumenteras genom att ta så många bilder vi kan i sådan hög frekvens som möjligt.
  \\ \hline
  Motivering &
Applikationsutvecklaren måste få möjlighet att se hur användaren navigerar i applikationen för att se hur användarvänligt gränssnittet är.
  \\ \hline
  Behov &
Minimum
  \\ \hline
  Prioritet &  
Skärminspelningen har hög prioritet. Det är en av de viktigaste funktionerna för att få reda på hur användaren använder appen. En snabb prototyp behövs men skärminspelningsarbetet är tidsödande och behöver arbetas med konstant för att få ett så bra resultat som möjligt.
  \\ \hline
  Källa &
Dokument med krav från The Beta Family.
  \\ \hline
  Verifiering &
Kravet får verifieras genom att testa produkten på ett antal olika applikationer för att se att det fungerar att spela in skärmen oavsett vilken app som använder den.
  \\ \hline
\end{tabular}

\subsubsection{Röstinspelning}
\begin{tabular}{ | p{65pt} | p{300pt} |}
  \hline
  Identifikator &
Röstinspelning
  \\ \hline
  Beskrivning & 
  Vad användaren säger under testningen av applikationen ska spelas in.
  \\ \hline
  Motivering &
  Att höra användarens tankegångar under testning utav användargränssnitt är högst önskvärt av apputvecklare.
  \\ \hline
  Behov &
  Minimum
  \\ \hline
  Prioritet &
  Röstinspelningen har hög prioritet. Det är en av huvudfunktionerna och en av de funktioner som har tidigast deadline.
  \\ \hline
  Källa &
  Dokument med krav från The Beta Family.
  \\ \hline
  Verifiering &
  Kravet får verifieras genom att testa produkten på ett antal olika applikationer för att se att det fungerar att spela in ljud oavsett vilken app som använder den.
  \\ \hline
\end{tabular}

\subsubsection{Kamerainspelning}
\begin{tabular}{ | p{65pt} | p{300pt} |}
  \hline
  Identifikator &
  Kamerainspelning
  \\ \hline
  Beskrivning & 
  Användarens ansikte ska spelas in under testningen av applikationen
  \\ \hline
  Motivering &
  En visuell bild av användaren under testningen kan ge ytterligare information kring hur denna finner användarvänligheten.
  \\ \hline
  Behov &
  Minimum
  \\ \hline
  Prioritet &
  Kamerainspelningen har hög prioritet. Det är en av huvudfunktionerna och en av de funktioner som hr tidigast deadline.
  \\ \hline
  Källa &
  Dokument med krav från The Beta Family
  \\ \hline
  Verifiering &
  Kravet får verifieras genom att testa produkten på ett antal olika applikationer för att se att det fungerar att spela in från kameran oavsett vilken app som använder den.
  \\ \hline
\end{tabular}

\subsubsection{Användaren har själv kontroll över vad som spelas in}
\begin{tabular}{ | p{65pt} | p{300pt} |}
  \hline
  Identifikator &
  Användarkontroll
  \\ \hline
  Beskrivning & 
  När en person betatestar en app med hjälp utav The Beta Familys ScreenRecorder ska de kunna välja själv vad för information de vill ge utvecklarna. Testaren ska själv kunna bestämma om frontkameran och mikrofonen ska spela in under testet. 
  \\ \hline
  Motivering &
  Testaren bör kunna stänga av vissa funktioner då man inte alltid befinner sig i situationer då man är bekväm att t.ex. tala högljutt om vad som händer i appen. Det kan handla om att man gör ett test på tunnelbanan på vägen hem och bara vill visa hur man navigerar runt i applikationen.
  \\ \hline
  Behov &
  Standard
  \\ \hline
  Prioritet &
  Att användaren själv kan stänga av vissa funktioner är önskvärt men inte av högsta prioritet.
  \\ \hline
  Källa &
  Dokument med krav från The Beta Family
  \\ \hline
  Verifiering &
  En oberoende part får testa att genomföra ett test med olika funktioner avstängda. Lyckas personen utan problem stänga av önskade funktioner kan kravet anses uppfyllt och dessutom är det ett tecken på bra interaktionsdesign.
  \\ \hline
\end{tabular}

\subsection{Tekniska krav och begränsningar}
\subsubsection{Snabb implementation av SDK:n}
\begin{tabular}{ | p{65pt} | p{300pt} |}
  \hline
  Identifikator &
  Snabb SDK-implementation.
  \\ \hline
  Beskrivning & 
  Enligt The Beta Family kan deras ScreenRecorder för iOS implementeras på under två minuter av utvecklaren. Det är önskvärt att Androidversionen är lika simpel att implementera.
  \\ \hline
  Motivering &
  En del av The Beta Familys affärsidé är att utvecklarna själva implementerar deras ScreenRecorder. Detta ställer krav på utvecklaren att de kan implementera bibliotek och skriva några rader kod. Därför vill The Beta Family att implementationen är så simpel som möjligt. Krävs en mängd invecklade operationer och beslut kan detta skrämma bort eventuella kunder som då kanske väljer att inte betatesta applikationen.
  \\ \hline
  Behov &
  Standard
  \\ \hline
  Prioritet &
  Att SDK:n är simpel att genomföra har ganska hög prioritet. Det viktigaste är självklart att den funktionsmässigt fungerar. Om den är krånglig att implementera kan den fortfarande användas av mer tekniska utvecklare. En SDK som är simpel att implementera kommer leda till att fler utvecklare får värdefull återkoppling och väljer att använda The Beta Familys tjänst igen.
  \\ \hline
  Källa &
  Uppstartsmöte med The Beta Family där VD:n Axel Nordenström gick igenom krav.
  \\ \hline
  Verifiering &
  När projektet är färdigställts kan en oberoende part låtas implementera SDK:n på tid. Lyckas personen implementera SDK:n på under två minuter kan detta krav ses som uppfyllt.
  \\ \hline
\end{tabular}

\subsubsection{Testning ska kunna ske var som helst}
\begin{tabular}{ | p{65pt} | p{300pt} |}
  \hline
  Identifikator & 
  Platsoberoende tester
  \\ \hline
  Beskrivning & 
  Ett test ska kunna genomföras var som helst. Det ska inte krävas att mobilen är kopplad till en dator eller att det finns internetuppkoppling.
  \\ \hline
  Motivering &
  Genom att möjliggöra tester oberoende av plats och internetuppkoppling fås fler tester. En testare kan t.ex. genomföra ett test på tåget hem från jobbet. Utökar man mängden tillfällen ett test kan ske på ökar man också sannolikheten att testaren gör ett väl genomfört test då det sker på testarens villkor. Dessutom är detta ett viktigt krav för applikationer som beror på platsen. En applikation för Stockholms kommunaltrafik skulle inte kunna testas om det fanns krav på att mobilen var kopplad till datorn via sladd.
  \\ \hline
  Behov &
Minimum
  \\ \hline
  Prioritet &
Prioriteten är hög, detta är något som samtliga gruppmedlemmar måste ha i åtanke från första början av utvecklingen. Android tillåter en del extra funktioner om mobilen befinner sig i felsökningsläge och är kopplad till datorn via sladd. På grund av kravet på platsoberoende testning är det viktigt att utvecklarna redan från början tänker på att inte använda sig av dessa funktioner. Det kan vara svårt, om inte omöjligt, att anpassa sig efter kravet i efterhand.
  \\ \hline
  Källa &
 Uppstartsmöte med The Beta Family där VD:n Axel Nordenström gick igenom krav.
  \\ \hline
  Verifiering &
  Om ett test kan genomföras på en mobil med avstängd datatrafik, utan att vara kopplad till en dator, kan kravet anses vara uppfyllt.
  \\ \hline
\end{tabular}

\begin{tabular}{ | p{65pt} | p{300pt} |}
  \hline
  Identifikator &
  \\ \hline
  Beskrivning & 
  \\ \hline
  Motivering &
  \\ \hline
  Behov &
  \\ \hline
  Prioritet &
  \\ \hline
  Källa &
  \\ \hline
  Verifiering &
  \\ \hline
\end{tabular}
