\subsection{Antaganden och beroenden}
\label{subsec:assumptions}

\subsubsection{Plattformsomfattning}
\label{subsubsec:plattformsomfattning}
TBF har idag inga krav på vilka versioner av iOS eller Android testbeställare får rikta sig till \parencite{betafaq}. iOS-versionen av SuperRecorder, som gruppens produkt strävar efter att likna, stödjer också alla versioner av iOS \parencite{superrec}. På grund av denna information har gruppen gjort antagandet att ett brett versionsstöd är en viktig punkt för TBF och dess kunder. Testbeställare som använder sig av SuperRecorder kommer att vänta sig att tjänsten fungerar på nya såväl som gamla versioner av Android, eftersom övriga delar av TBF:s testtjänster redan gör det. Versionsoberoende lösningar bör därför alltid föredras framför tillämpningar som baserar sig på ny systemfunktionalitet.

\subsubsection{Testbeställarens målgrupp}
Apputvecklare söker sig till TBF för att testa sina produkter på ett så antal scenarion som möjligt. Förutom att nå ut till så många plattformar som möjligt, se \ref{subsubsec:plattformsomfattning}, vill testbeställare givetvis nå ut till så många olika användare som möjligt. Detta innebär idealt att testarna som TBF förmedlar ska ha en brett kompetensspann, med både ovana och avancerade användare inkluderat. För att kunna nå ut till en så varierande skara användare bör SuperRecorder ställa så få krav på användarna som möjligt. Tjänsten bör likt SuperRecorder för iOS kunna användas utan speciella förkunskaper, och utan speciella krav på befintlig hård- och mjukvara, så länge testarens mobil är baserad på Android.

\subsubsection{Testarens nätanslutning}
För att SuperRecorder ska kunna ge en högupplöst gestaltning av ett användarscenario måste tjänsten skicka relativt mycket data mellan testarens mobil och TBF:s servrar. Detta är i synnerhet sant om behandlingen av rådatan sker på server-sidan, vilket kan bli ett krav för att kunna nå ut till testare med äldre mobiltelefoner. För att kunna skapa en produkt som motsvarar TBF:s förväntningar måste vi därför anta att testaren har en robust och snabb nätanslutning. Allra helst ska testaren vara WiFi-ansluten vid testtillfället.

\subsubsection{Testbeställarens tillgång till källkod}
Integreringen av SuperRecorder för iOS sker genom att importera TBF:s bibliotek till applikationen, och sedan lägga till ett par rader i källkoden \parencite{superrec}. Detta är en hyfsat trivial handling i utvecklingsverktyget för iOS. Gruppens efterforskningar har visat att motsvarande integrering i Android troligen inte är lika smärtfri. Ändringar kan komma att bli nödvändiga på ett flertal ställen i testbeställarens källkod för att SuperRecorder ska få önskad funktionalitet. Denna process har goda möjligheter för att kunna automatiserats med program, men kräver dock att testbeställaren har full tillgång till hela den testade applikationens källkod. Använder sig den testade applikationen av redan kompilerade tredjeparts-bibliotek, kommer dessa moduler inte att kunna loggas med SuperRecorder.

\subsubsection{Serversidans miljö}
\label{subsubsec:serverside}
Det finns ett flertal öppna och effektiva program och bibliotek för behandling av bitmaps-bilder och videoskapande av dessa. För att kunna nyttja dessa är det dock nödvändigt att servermiljön som SuperRecorder anropar har möjlighet att använda sig av godtycklig programvara. Gruppen antar därför att TBF har tillgång till en servermiljö som de kan modifiera efter eget tycke, och som alltså inte är låst till en viss mängd fördefinierade verktyg.
