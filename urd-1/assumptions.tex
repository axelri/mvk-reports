\subsection{Antaganden och beroenden}
\subsubsection{Plattformsomfattning}
\label{subsubsec:plattformsomfattning}
TBF har idag inga krav på vilka versioner av iOS eller Android testbeställare får rikta sig till \parencite{betafaq}. iOS-versionen av SuperRecorder, som gruppens produkt strävar efter att likna, stödjer också alla versioner av iOS \parencite{superrec}. På grund av denna information har gruppen gjort antagandet att ett brett versionsstöd är en viktig punkt för TBF och dess kunder. Testbeställare som använder sig av SuperRecorder kommer att vänta sig att tjänsten fungerar på nya såväl som gamla versioner av Android, eftersom övriga delar av TBF:s testtjänster redan gör det. Versionsoberoende lösningar bör därför alltid föredras framför tillämpningar som baserar sig på ny systemfunktionalitet.

\subsubsection{Testbeställarens målgrupp}
Apputvecklare söker sig till TBF för att testa sina produkter på ett så antal scenarion som möjligt. Förutom att nå ut till så många plattformar som möjligt, se \ref{subsubsec:plattformsomfattning}, vill testbeställare givetvis nå ut till så många olika användare som möjligt. Detta innebär idealt att testarna som TBF förmedlar ska ha en brett kompetensspann, med både ovana och avancerade användare inkluderat. För att kunna nå ut till en så varierande skara användare bör SuperRecorder ställa så få krav på användarna som möjligt. Tjänsten bör likt SuperRecorder för iOS kunna användas utan speciella förkunskaper, och utan speciella krav på befintlig hård- och mjukvara, så länge testarens mobil är baserad på Android.

\subsubsection{Testarens nätanslutning}
För att SuperRecorder ska kunna ge en högupplöst gestaltning av ett användarscenario måste tjänsten skicka relativt mycket data mellan testarens mobil och TBF:s servrar. Detta är i synnerhet sant om behandlingen av rådatan sker på server-sidan, vilket kan bli ett krav för att kunna nå ut till testare med äldre mobiltelefoner. För att kunna skapa en produkt som motsvarar TBF:s förväntningar måste vi därför anta att testaren har en robust och snabb nätanslutning. Allra helst ska testaren vara WiFi-ansluten vid testtillfället.
