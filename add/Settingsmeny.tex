\subsubsection{Typ}
Java klass och xml-dokument

\subsubsection{Syfte}
URD-krav 4
URD-krav 6
Visualiserar navigationsmöjligheter i GUI;t och tillhandahåller för användaren möjlighet att välja vad som ska spelas in under en testsession. 

\subsubsection{Funktionalitet}
Settingsmenyn består av knappar, checkboxes och en färgväljare som skickar information om att de tryckts till GUI vilket sedan utför de händelser varje inputenhet ska utlösa.

\subsubsection{Underordnade Komponenter}
Colorslider.

\subsubsection{Beroenden}
GUI;t måste visas för att denna komponent ska vara relevant.

\subsubsection{Gränssnitt}
Vid tryckning på given inputenhet skickas information om att den specifika inputenheten blivit tryckt till GUI.

\subsubsection{Resurser}
Settingsmenyn behöver GUI för att fungera.

\subsubsection{Referenser}
Settingsmenyn är implementerad i Android och därav behövs kunskap om Android för att förstå komponenten.
http://developer.android.com/reference/packages.html

\subsubsection{Processer}
För varje inputenhet tillskrivs följande listener:
If tryckt
then gui.inputenhetused(denna_inputenhet)

\subsubsection{Data}
Settingsmenyn lagrar ingen intern data förutom det som följer genom användandet av Androids biblotek.