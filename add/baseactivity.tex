\subsubsection{Typ}
Java-klass
\subsubsection{Syfte}
Syftet med denna klass är att vara en mellanhand mellan applikationernas aktiviteter och klassen Activity. Detta behövs för att ta reda på vilken Activity som användaren befinner sig i under körning.
\subsubsection{Funktionalitet}
Klassen är menad att användas som superklass i varje aktivitet. När en aktivitet startas kallas basaktivitetens funktion onResume() som sparar undan en referens till aktiviteten. När en aktivitet avslutas kallas basaktvitetens funktion onDestroy() som i sin tur sätter referensen till null. Sistnämnda är mycket viktigt för att undvika massiva minnesläckor.
\subsubsection{Underordnade komponenter}
Alla aktiviteter till applikationen är underordnade komponenter. 
\subsubsection{Beroenden}
Inga beroenden
\subsubsection{Gränssnitt}
Denna komponent har inget direkt gränssnitt förutom den sparade referensen. Rörelse- och skärminspelning beror på den sparade referensen.
\subsubsection{Resurser}
Klassen implementerar Activity från Android SDK.
\subsubsection{Referenser}
Grundläggande kunskap om Android underlättar förståelsen av kompontenten. Se http://developer.android.com
\subsubsection{Process}
Subkomponenterna, det vill säga testapplikationens aktiviteter, kallar på super.onResume() startar och super.onDestroy() i sina egna onResume() och onDestroy() funktioner. Dessa kallas när en aktivitet startar respektive avslutas. Vår basaktivitets kod kommer då att köras och kommer se ut ungefär som nedan.
\begin{verbatim}
protected void onResume() {
		set currentActivity to thisActivity;
}
protected void ondestroy() {
		set currentActivity to null;
}
\end{verbatim}
\subsubsection{Data}
 Komponenten har inga lokala data eller strukturer.