\subsubsection{Typ}
Java klass och xml-dokument

\subsubsection{Syfte}
URD-krav 6 \\
Visualiserar inloggningsmöjligheter i GUI:t. Den första menyn som visas vid visningen av GUI:t första gången. 

\subsubsection{Funktionalitet}
Inloggningsmenyn består av knappar, textfält för användarnamn samt lösenord, en knapp ska motsvara att en inloggningsrequest ska skickas och den andra att inloggningen skippas.

\subsubsection{Underordnade Komponenter}
Subkomponenter saknas.

\subsubsection{Beroenden}
GUI:t måste visas för att denna komponent ska vara relevant samt bör inloggningsservern vara i drift.

\subsubsection{Gränssnitt}
Vid tryckning på given knapp skickas information om att den specifika knappen tryckts till GUI:t.

\subsubsection{Resurser}
Inloggningsmenyn behöver GUI för att fungera.

\subsubsection{Referenser}
Inloggningsmenyn är implementerad i Android och därav behövs kunskap om Android för att förstå komponenten.
\url{http://developer.android.com/reference/packages.html}

\subsubsection{Processer}
För varje inputenhet tillskrivs följande listener:
\begin{verbatim}
If tryckt
then gui.inputenhetused(denna_inputenhet)
\end{verbatim}

\subsubsection{Data}
Settingsmenyn lagrar ingen intern data förutom det som följer genom användandet av Androids biblotek.