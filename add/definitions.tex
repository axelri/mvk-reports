\subsection{Begreppsdefinitioner}
\label{subsec:definitions}

\paragraph{IP} 
Internetprotokoll, ett protokoll för att överföra data mellan olika nätverk. Innehåller bland annat en IP-adress vilket informerar om vart datan ska skickas. \parencite{inetF1}

\paragraph{TCP}
Transmission Control Protocol, ett dataöverföringsprotokoll som upprättar en förbindelse och ser till att datan kommer fram säkert. \parencite{inetF1}

\paragraph{HTTP}
Hypertext Transfer Protocol, ett förfrågan-svar-protokoll för en klient-server-modell. Med denna protokoll kan en klient göra förfrågan till en server, som i sin tur har möjligheten att svara på denna förfrågan. \parencite{http}

\paragraph{CRUD}
Create, Read, Update and Delete. De fyra funktioner som krävs för grundläggande persistent lagring av data. Motsvarar HTTP request-metoderna (i ordning) POST, GET, PUT, DELETE.