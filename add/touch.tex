\subsubsection{Typ}
En metod som skriver rörelsehändelserna till en fil.
\subsubsection{Syfte}
På serversidan kommer olika delar kombineras för att generera en video som visar användarens test av applikationen. Rörelsehändelserna kommer representeras i form utav en text-fil med data. När videon sedan renderas kommer denna data användas för att rita ut rörelsehändelserna.
\subsubsection{Funktionalitet}
När en inspelning startar börjar mobilen spara varje rörelsehändelse. Denna information sparas till en fil enligt JSON-standard. Exempel på den information som sparas är: \\
\begin{verbatim}
{timestamp=''1000'', action=''up'', index=''0'', x=''310'', y=''670''}
 \end{verbatim} 
 \textbf{Timestamp} är tiden i millisekunder sedan inspelningen startade. Detta används för att kunna synka rörelsehändelserna med videon.
 
 \textbf{Action} är vilken typ av rörelsehändelse det är. De tre vanligaste typerna är ``up'' när man släpper upp fingret, ``down'' när man placerar fingret på skärmen och ``move'' när man rör fingret över skärmen.
 
 \textbf{Index} är vilket finger på skärmen som står för rörelsehändelsen. $Index=0$ är det fingret som först nuddade skärmen, $index=1$ är andra fingret som nuddade skärmen, o.s.v. Detta gör det möjligt att rita ut rörelser som kommer från flera olika fingrar utan att blanda ihop dem.
 
 \textbf{X och Y} är den position fingret har i X- och Y-led.
\subsubsection{Underordnade komponenter}

\subsubsection{Beroenden}
Rörelsehändelserna kan skrivas oberoende av andra komponenter. För använda informationen behövs dock en överföringsdel. Säkerheten i Android gör att inga utomstående applikationer eller användare har tillgång till filen som skrivits. 
\subsubsection{Gränssnitt}

\subsubsection{Resurser}

\subsubsection{Referenser}

\subsubsection{Process}

\subsubsection{Data}
