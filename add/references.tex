\subsection{Referenser}

\subsubsection{Developer Android}
\url{http://developer.android.com} \\
Vill man lära sig mer om Android och hur det fungerar så är Googles egna hemsida för androidutvecklare väldigt bra. Där finns väldigt mycket information om olika metoders betydelse och användning. Här hittas den grundläggande informationen om biblioteket för Android. Nedan listas ett antal av de områden vi har utforskat hittills i projektet.

\url{http://developer.android.com/guide/components/fundamentals.html} \\
Information om den grundläggande funktionaliteten i Android.

\url{http://developer.android.com/about/versions/kitkat.html#44-media} \\
Vad som gäller för media i den senaste versionen utav Android. Video-, mikrofon- och skärminspelning kan man hitta mer information om här.

\url{http://developer.android.com/reference/packages.html} \\
Information om de många olika API:s som finns till Android.

\url{http://developer.android.com/about/dashboards/index.html?utm_source=ausdroid.net} \\
Information om de olika enheter som använder sig av Android och hur de är fördelade över olika versionerna.

\url{http://developer.android.com/guide/topics/media/audio-capture.html} \\
Information om ljudinspelning.

\url{http://developer.android.com/reference/android/media/MediaRecorder.html} \\
Inspelning av media i olika format.

\url{http://developer.android.com/guide/topics/ui/dialogs.html} \\
Här hittas information om hur dialogrutor fungerar och implementeras. Detta är rutor som kommer upp på skärmen när användaren måste göra ett val eller ändra information.

\url{http://developer.android.com/reference/android/view/MotionEvent.html} \\
Information om hur man rapporterar användarens rörelser på skärmen.

\url{http://developer.android.com/reference/java/net/HttpURLConnection.html} \\
Klassen som används för att skapa en förbindelse med en server och skicka data med HTTP-protokollet.

\subsubsection{Laravel}
\url{http://laravel.com} \\
Förutsatt att man kan objektorienterad PHP så är Laravel ett smidigt ramverk för att snabbt skapa skalbara webapplikationer.

\subsubsection{Zurb}
\url{http://zurb.com/home} \\
Här man läsa mer om designföretaget Zurb som utvecklar ett frontend-ramverk för webgränssnitt.

\subsubsection{SCR ScreenRecorder på Google Play}
\url{https://play.google.com/store/apps/details?id=com.iwobanas.screenrecorder.free}

\subsubsection{ScreenRecorder på Google Play}
\url{https://play.google.com/store/apps/details?id=com.nll.screenrecorder}
\url{https://android-review.googlesource.com/\#/c/8866/}

\subsubsection{The Application Sandbox}
\url{http://source.android.com/devices/tech/security/\#the-application-sandbox}
Här kan man lära sig mer om varför det inte går att spela in information från tredjepartsapplikationer.

\subsubsection{Introduktion till Datalogi, DD1339}
\url{http://www.kth.se/student/kurser/kurs/DD1339} \\
Kurshemsidan till Introduktion till Datalogi DD1339 på KTH.

\subsubsection{The Beta Family, SuperRecorder}
\url{http://thebetafamily.com/superrecorder} \\
SuperRecorder för iOS kan ses via länken och det är gruppens mål att göra en liknande SDK till Android. Genom att studera iOS-versionen kan information hittas om den önskade funktionaliuteten.

\subsubsection{Vanliga frågor om betatestning med SuperRecorder}
\url{http://thebetafamily.com/faq/for-developers\#which-mobile-platforms-do-you-support-for-beta-testing} \\
Vanliga frågor angående beta-tester av applikationer med SuperRecorder.

\subsubsection{Projektkatalog}
\url{http://www.nada.kth.se/~karlm/mvk/mvk13/ProjectCatalog_DD1392.pdf} \\
Här hittas alla tillgängliga projekt i kursen.

\subsubsection{FFmpeg}
\url{http://www.ffmpeg.org/} \\
Information om verktyget FFmpeg som hanterar audio och video.

\subsubsection{Skapandet av mediafiler i formatet FFmpeg}
\url{http://www.roman10.net/how-to-build-ffmpeg-with-ndk-r9/} \\
Information om hur man skapar mediafiler med verktyget FFmpeg med NDK r9.

\subsubsection{Databasteknik, DD1368}
\url{http://www.csc.kth.se/utbildning/kth/kurser/DD1368/} \\
Kurshemsidan till Databasteknik DD1368 på KTH.

\subsubsection{Dubbeltryck med två fingrar}
\url{http://stackoverflow.com/questions/12414680/how-to-implement-a-two-finger-double-click-in-android} \\
Här kan man läsa mer om hur man implementerar dubbeltryck med två fingrar på skärmen.

\subsubsection{LookBack}
\url{http://lookback.io} \\
LookBack är en av The Beta Familys konkurrenter. På deras hemsida kan man se hur de har skapat sin produkt, hur den skiljer sig och vad den har för likheter med projektets produkt.

\subsubsection{Att vara utvecklare för Android}
\url{http://techcrunch.com/2012/05/11/this-is-what-developing-for-android-looks-like/} \\
Här kan man läsa mer om hur det ser ut att vara utvecklare för Android. Det finns väldigt många olika hårdvarukonfigurationer i Androidenheterna på marknaden, vilket gör att det blir svårt att utveckla applikationer som fungerar på alla.

\subsubsection{ImageMagick}
\url{http://www.magickwand.org/} \\
Denna referens visar hur man kan hantera bilder med ImageMagick i PHP.

\subsubsection{Personas}
Bilderna som används för personas ägs av Yuri Samoilov respektive Logan Campbell.
