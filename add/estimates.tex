\section{Genomförbarhetsstudie}
\label{sec:estimates}

\subsection{Minimum} % (fold)
\label{sub:Minimum}

Ett antal av de definierade komponenterna relaterar direkt till projektets ``minimum-behov''. Detta innebär att komponenterna är essentiella för att projektet ska uppnå utlovad funktionalitet. Komponenterna som faller inom denna kategori är \textit{skärminspelning}, \textit{rörelser} \textit{röstinspelning} och \textit{kamerainspelning}. Alla dessa har redan färdiga prototyper som fungerar väl på egen hand. Kvarvarande uppgift är därför att sätta ihop modulerna till ett system. Androids interna systembeteende har visat sig försvåra vissa typer av externa anrop. Exempelvis kan \textit{skärminspelning} i nuvarande form kollidera med Androids egna grafikkod och medföra krascher och/eller försämrad prestanda. Problem likt dessa har dock inte varit lika framträdande i övriga ``minimum-moduler''. Gruppen har därför goda skäl att tro att dessa moduler har möjlighet att fungera tillräckligt bra tillsammans för att önskvärd funktionalitet ska uppnås.

\subsection{Standard} % (fold)
\label{sub:Standard}

Komponenter som sträcker sig utanför de allra mest grundläggande kraven, men ändå är viktiga för en godtagbar användarupplevelse, faller inom ``standard-behov''. Komponenterna inom denna kategori är övriga definierade komponenter, utom de som redan blivit indelade i ``minimum"-kategorin''. Även många komponenter i ``standard''-kategorin har färdiga prototyper som har visat sig fungera väl i avskiljda testmiljöer. Det är inom denna kategori dock inte lika självklart att komponenter fungerar bra nog tillsammans för att uppnå tillräcklig prestanda. Speciellt projektets komponenter på server-sidan, \textit{databehandling} med \textit{rörelsegestaltning} och \textit{skärmgestaltning}, utför en tyngre sorts beräkningar som troligen behöver optimeras från den nuvarande prototyp-formen. Uppskattningsvis tar sammanställningen av en video av längd $n$ sekunder mellan $0,{}75n$ och $n$ sekunder på modern serverhårdvara. Detta kan vara ohållbart i en storskalig miljö med flera simultana användare.

Värt att nämna är att dessa uppskattningar baserar sig på körning av enklare prototypmoduler, som har utvecklats av ett fåtal gruppmedlemmar och med låg arbetsprioritet. Då den ursprungliga projektplanen i PPD-1 har följts med avseende på mjukvaru-utveckling har flera gruppmedlemmar frigjorts för att kunna arbeta tillsammans på de komplexare ``standard-modulerna''. Detta bådar gott för att den dedikerade utvecklingstiden till dessa komponenter ska räcka för att optimera körningen tillräckligt.

\subsection{Övrigt} % (fold)
\label{sub:Ovrigt}

Komponenten \textit{kodintegration} tillför inte projektet någon ny funktionalitet under körning, utan syftar endast till att underlätta integrationen mellan SuperRecorder SDK och testbeställarens egen Android-applikation. Komponenten har därför tillägnats en låg prioritet, även om den direkt är kopplad till ett ursprungligt krav från projektbeställarna. Gruppens befintliga research i frågan tyder på att en enklare version som fungerar för majoriteten av fallen är klart genomförbar inom tidsspannet för utvecklingsfasen. Det är däremot oklart ifall tid finns för att testa och utvidga komponenten till en mera robust lösning. Ett alternativ är att erbjuda en prototyp av komponenten till slutkund, för att senare hänvisa dessa till manuell integrering om komponenten inte fungerar för just deras fall.

% subsection Integration (end)

\subsection{Tidsplan}
\begin{figure}[H]
\centering
\begin{tabular}{ | l | l | l |}
  \hline
  \textbf{Systemmodul} & \textbf{Deadline} & \textbf{Uppskattad tidsåtgång} \\ \hline
  GUI & 6/4 & 30h \\ \hline
  GUI Support & 6/4 & 10h \\ \hline
  Inloggningsmeny & 6/4 & 10h \\ \hline
  Huvudmeny & 6/4 & 10h \\ \hline
  Settingsmeny & 6/4 & 5h \\ \hline
  Färgreglage & 6/4 & 5h \\ \hline
  Rörelser & 6/4 & 10h \\ \hline
  Rörelsegestaltning & 6/4 & 30h \\ \hline
  Kamerainspelning & 6/4 & 20h \\ \hline
  Ljudinspelning & 6/4 & 5h \\ \hline
  Videomeny & 4/5 & 10h \\ \hline
  Videolista & 4/5 & 40h \\ \hline
  Basaktivitet & 4/5 & 50h \\ \hline
  Databehandling & 4/5 & 50h \\ \hline
  Skärminspelning & 4/5 & 300h  \\ \hline
  Skärmgestaltning & 4/5 & 300h \\ \hline
  Komprimering & 4/5 & 200h \\ \hline
  Överföring & 4/5 & 200h \\ \hline
\end{tabular}
\caption*{\textit{Komponent-deadlines}}
\end{figure}
