\subsection{Överblick över dokumentet}

\nameref{sec:system_overview} summerar systemkontexten och systemdesignen som diskuteras utförligare under sektionerna \nameref{sec:system_context} och \nameref{sec:system_design}. \nameref{sec:system_context} ger den detaljerad beskrivning av systemkontexten (servern och mobilapplikationer) genom illustrationer i form av UML-diagram. \nameref{sec:system_design} ger en överblick över designmetoden \textit{Rational Unified Process} som används.

\nameref{sec:component} innehåller detaljerad information om alla komponenter i systemet. Varje komponent beskrivs enligt: typ, syfte, funktionalitet, underordnade komponenter, beroende, gränssnitt, resurser, referenser, process och data.

\nameref{sec:estimates} summerar slutsatserna kring genomförbarheten av projektet. Komponenterna i systemet tilldelas en prioritet: minimum eller standard. Komponenter som \textit{skärminspelning}, \textit{rörelseinspelning} med flera är ett krav för projektet. Dessa komponenter prioriteras före komponenter som inte är absolut nödvändiga.

\nameref{sec:user_req} innehåller en tabell som kryssrefererar komponenter med användarbehov som beskrevs i URD-1.