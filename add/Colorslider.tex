\subsubsection{Typ}
Java-klass och xml-dokument

\subsubsection{Syfte}
URD-krav 4 \\
URD-krav 6 \\
Möjliggöra för användaren att välja färg för visualiseringen av tryckningar för testsessioner.

\subsubsection{Funktionalitet}
Färgreglaget består av en utritad färggradient. Vid tryckning på denna gradient kommer den färg som trycktes att skickas vidare till GUI:t.

\subsubsection{Underordnade Komponenter}
Subkomponenter saknas.

\subsubsection{Beroenden}
GUI:t måste visas för att denna komponent ska vara relevant.

\subsubsection{Gränssnitt}
Genom att lägga till en listener till färgreglaget kommer denna komponent att skicka information om vilken färg som tryckts på till alla listeners. 

\subsubsection{Resurser}
Det behövs något objekt som lagt till en listener till färgreglaget.

\subsubsection{Referenser}
Inga externa dokument är nödvändiga.

\subsubsection{Processer}
Vid skapandet räknas en färggradient ut baserat på längden på slidern och lagras. Vid tryckning på färgreglaget hämtas sedan färgen ut för det tryckta x-värdet och skickas till alla listeners.

\subsubsection{Data}
En lokal färggradient lagras.