\subsubsection{Typ}
Servermodul som består av skriptfiler och kompilerad programkod.

\subsubsection{Syfte}
Syftet med rörelsegestaltningen är att skapa en video som gestaltare användarens rörelsemönster under testningen av Android-applikationen.

\subsubsection{Funktionalitet}
Loggar med användarens olika fingerrörelser genom testningen skickas från komponenten \nameref{sub:dataprocessing}. Loggarna används sedan till att rita upp ett rörelseflöde i en video, som ska motsvara användarens faktiska fingerrörelser.

\subsubsection{Underordnade komponenter}
Denna komponent har inga underordnade komponenter.

\subsubsection{Beroenden}
Beronden för komponenten är data från den testade Android-applikationen, samt en servermiljö där \textit{ImageMagick}, \textit{FFmpeg} och tillhörande kodbibliotek finns tillgängligt och tillhörande kodbibliotek finns tillgängligt.

\subsubsection{Gränssnitt}
Komponenten anropas med loggar för en applikationstestning. Komponenten matar sedan ut en video med fingerrörelserna gestaltade.

\subsubsection{Resurser}
Modulen kräver en servermiljö med \textit{ImageMagick}, \textit{FFmpeg} och tillhörande kodbibliotek tillgängligt.

\subsubsection{Referenser}

\subsubsection{Process}
Modulen går igenom varje rörelserelaterad händelse i tur och ordning. För varje händelse ritas först händelsen upp i en bildruta med hjälp av bildbehandlings-biblioteket \textit{ImageMagick}. När händelsen finns representerad i en bitmap används \textit{FFmpeg} för att rita upp så många rutor som händelsen ska visas i filmen. Detta moment är mycket snarlikt processen i \nameref{sub:screenmedia}. När alla händelser har gåtts igenom skickas den slutgiltiga filmen ut. Med tillräckligt högupplösta loggar skapar processen en video med gott flöde.

\subsubsection{Data}
Modulen använder sig av två långlivade datastrukturer; en videobuffer och en kö med indata från loggarna. Varje element i indata behandlas och skapar en ny temporär datastruktur, en bitmap. När denna bitmap har skrivits korrekt antal gånger till slutet av videobuffern raderas den. När hela kön har gåtts igenom skrivs videobuffern över till hårddisken och processen är slutförd.
