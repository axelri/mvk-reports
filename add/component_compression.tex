\subsubsection{Typ}
En Java-klass som tar en eller flera filer och komprimerar samt arkiverar dem till en enda zip-fil. 

\subsubsection{Syfte}
När en betatestinspelning är slutförd genereras en mängd data som måste överföras till en server där den sammanställs för att sedan kunna presenteras för applikationsutvecklarna. Det är mer praktiskt att få all denna information i ett enda paket än att få alla filer skickade var för sig. Ju mindre filstorleken är desto mindre tid tar överföringen och mindre bandbredd används. Allt detta leder till att komprimering av data till en enda zip-fil utförs innan överföringen till servern sker. Detta relaterar till punkt 2.3.5 (sammansättning av media på serversidan) samt 2.3.6 (överföring) i URD:n. 

\subsubsection{Funktionalitet}
Klassen implementerar en publik metod \verb:zipFolder(): som tar två parametrar: en sökväg till mappen som ska komprimeras samt en sökväg till den komprimerade filen. Metoden går då rekursivt igenom alla filer och mappar, komprimerar dem och lägger till dem i en zip-fil. 

\subsubsection{Underordnade komponenter}
Inga underordnade komponenter

\subsubsection{Beroenden}
För att att klassen ska kunna komprimera filer krävs att filerna existerar samt att sökvägarna till filerna stämmer. 

\subsubsection{Gränssnitt}
Efter en slutförd inspelning av betatestning av en applikation genereras en mängd data som skrivs till filer. Dessa filer består av bland annat alla skärmdumpar som sedan ska sammanställas till en video, information om var användaren har tryckt på skärmen, videoinspelning av användarens ansikte, ljudinspelning av användaren med mera. Innan all data skickas till servern används denna klass för att komprimera filerna till en enda fil, som sedan skickas till servern. 

\subsubsection{Resurser}
Lagringsutrymme krävs för den genererade zip-filen. 

\subsubsection{Referenser}
Att lagra datan i flertalet filer i en enda, mindre fil kräver två steg: komprimering och arkivering. Komprimering leder till att en fil ockuperar mindre minnesutrymme (i detta fall på ett ickedestruktivt sätt, det vill säga det går att återskapa den ursprungliga datan utifrån den komprimerade datan) och arkivering innebär att flera filer lagras i en enda fil. \href{http://en.wikipedia.org/wiki/Gzip}{Gzip} är ett annat alternativ som ofta används vid komprimering, framförallt i Linuxmiljöer, vilket generar filer med filändelsen \verb:.gz:. Nackdelen med denna är att gzip inte stödjer arkivering, vilket oftast görs med programmet tar, och således är det vanligt att se komprimerade filer med filändelsen \verb:.tar.gz:. 

Zip stödjer både komprimering och arkivering, vilket är anledningen till att denna används. 

\subsubsection{Process}
Pseudokod: 
\begin{verbatim}
Class FileCompression {

	private constructor for non-instantiability

	public static  void zipFolder(String sourceFolderPath, String zipFilePath){
		Open outputStream
		addFolder(outputStream, sourceFolder, sourceFolder)
		Close outputStream
	}
	
	private static void addFolder(ZipOutputStream zos, String thisFilePath, String baseFolderPath){
		File f = new File(thisFilePath)
		
		if(f is a folder){
			if(f is a subfolder of baseFolder){
				add f as a folder entry in the zip-file, keeping the folder structure with root in baseFolder
				// Example: thisFilePath == ".../baseFolder/f1/f2/thisFolder"
				// In the zip-file, the folder entry /f1/f2/thisFolder/ will be created
			}
			for(every subfile in folder f){
				addFolder(zos, subfile, baseFolderPath)
			}
		} else {
			// f is a file, not a folder
			add f as a file entry in the zip-file, keeping the folder structure with root i baseFolder
			// Example: thisFilePath == ".../baseFolder/f1/f2/thisFile"
			// In the zip-file, the file entry /f1/f2/thisFile will be created
		}

	}
}
\end{verbatim}

\subsubsection{Data}
Klassen skapar en dataström per fil för indata samt en buffer för zip-filen som utdata. Sökvägarna till filerna fås genom argumenten i metoden \verb:public static  void zipFolder(String sourceFolderPath, String zipFilePath):. För varje fil i indatan skapas en ny dataström som läser in filen och sedan skrivs fildatan till en speciell buffer för zip-filen: \verb:ZipOutputStream out:. För varje infil skapas även en \verb:ZipEntry entry: som specificerar var i zip-filens mappstruktur som den inlästa filen ska läggas. Den slutgiltiga zip-filen producerar ett arkiv med mappstrukturen:
\begin{description}
  \item[Screenshots] \hfill \\
  	Skärmdumpar.
  \item[Touch] \hfill \\
  	JSON-filer med information om användarens fingerrörelser.
  \item[Video] \hfill \\
  	Videoinspelning av användaren.
  \item[Video] \hfill \\
  	Röstinspelning av användaren.
\end{description}