\subsubsection{Typ}
Java klass och xml-dokument

\subsubsection{Syfte}
URD-krav 6
Visualiserar vilka inspelningar som finns lagrade i SuperRecorder.

\subsubsection{Funktionalitet}
Videolistan består av en scrollbar lista som visualiserar mängden av alla inspelningar som finns lagrade i Super Recorder. Det finns dessutom en knapp som ska motsvara att backa till huvudmenyn. Vid tryckning på ett element i listan skall detta begära en meny där detta objekt kan behandlas.

\subsubsection{Underordnade Komponenter}
Subkomponenter saknas.

\subsubsection{Beroenden}
GUI;t måste visas för att denna komponent ska vara relevant. Dessutom måste en pekare till den plats i minnet där videorna för testsessionerna finnas tillgänglig.

\subsubsection{Gränssnitt}
Vid tryckning på given inputenhet skickas information om att den specifika inputenheten tryckts till GUI;t. Ifall ett element i listan trycks så skickas informationen att det specifika elementet har trycks.

\subsubsection{Resurser}
Videolistan behöver GUI för att fungera.

\subsubsection{Referenser}
Videolistan är implementerad i Android och därav behövs kunskap om Android för att förstå komponenten.
http://developer.android.com/reference/packages.html

\subsubsection{Processer}
För varje inputenhet tillskrivs följande listener:
If tryckt
then gui.inputenhetused(denna_inputenhet)

\subsubsection{Data}
Videolistan lagrar ett Adapter-objekt som kan upptäcka om förändringar har skett i listan som lagrar inspelade testsessioner. Förutom det lagras ingen intern data förutom det som följer genom användandet av Androids biblotek.
